% !TeX spellcheck = en_US
\documentclass[]{ccs-thesis}
% options:
% [germanthesis] - Thesis is written in German
% [plainunnumbered] - Don't print numbers on plain pages
% [earlydraft] - Settings for quick draft printouts
% [watermark] - Print current time/date at bottom of each page
% [phdthesis] - switch to PhD thesis style
% [twoside] - double sided
% [cutmargins] - text body fills complete page


% Author name. Separate multiple authors with commas.
\author{Philip Frerk}
\birthday{March 31, 1994}
\birthplace{Bielefeld}

% Title of your thesis.
\title{Wireless Protection of Vulnerable Road Users}

% Choose one of the following lines. Feel free to change the word "Informatik" to match your degree program.
%\thesistype{Masterarbeit im Fach Informatik}\thesiscite{Master's Thesis~(Masterarbeit)}
%\thesistype{Bachelorarbeit im Fach Informatik}\thesiscite{Bachelor Thesis~(Bachelorarbeit)}
\thesistype{Seminar Thesis in Master's Computer Science}\thesiscite{Seminar Thesis~(Seminararbeit)}

% List of advisors, separated by commas.
\advisors{Christoph Sommer}

% List of referees, separated by commas.
\referees{Christoph Sommer, Falko Dressler}


% Define abbreviations used in the thesis here.
\acrodef{WSN}{Wireless Sensor Network}
\acrodef{MANET}{Mobile Ad Hoc Network}
\acrodef{ROI}{Region of Interest}{short-indefinite={an}, long-plural-form={Regions of Interest}}
\acrodef{ADAC}{German Automobile Association}{foreign={Allgemeiner Deutscher Automobilclub}}
\acrodef{CANhashing}[CAN]{Content Addressable Network}{extra={when referring to the distributed hash table}}
\acrodef{CANproto}[CAN]{Controller Area Network}{extra={when referring to the bus protocol}}
\acrodef{VU}{Vulnerable Road User}
\acrodef{V2PCS}{Vehicle to Pedestrian Communication System}
\acrodef{GDA}{Geographical Destination Area}
\acrodef{POC}{Probability of Collision}

\begin{document}

\pagenumbering{roman}

\maketitle

\chapter*{Abstract}
\addcontentsline{toc}{chapter}{Abstract}
\begin{otherlanguage*}{american}

about 1/2 page:
\begin{enumerate}
	\item Motivation (Why do we care?)
	\item Problem statement (What problem are we trying to solve?)
	\item Approach (How did we go about it)
	\item Results (What's the answer?)
	\item Conclusion (What are the implications of the answer?)
\end{enumerate}

Protecting \acp{VU} is a very important task as in roughly 50 \% of all traffic accidents \acp{VU} are involved. \acp{VU} are pedestrians or drivers of two-wheeled vehicles.

A technology is needed that warns both the \ac{VU} and the car driver if an accident between them is likely to happen.  This is a challenging task, because the warnings have to be sent in time  and also it has to ensured that no people are warned who are not really affected by the approaching car.

To achieve that goal, wireless networks, GPS and sensor perception will be used.

Results show that the number of accidents with \acp{VU} involved can be reduced dramatically.

Therefore, much more work will be put into this topic, because it is already shown that the technologies can potentially prevent many traffic accidents.

\end{otherlanguage*}


\chapter*{Kurzfassung}
\addcontentsline{toc}{chapter}{Kurzfassung}
\begin{otherlanguage*}{ngerman}

Der Schutz gefährdeter Verkehrsteilnehmer ist eine sehr wichtige Aufgabe, denn in etwa 50 \% aller Verkehrsunfälle sind gefährdete Verkehrsteilnehmer involviert. Gefährdete Verkehrsteilnehmer sind Fußgänger oder Fahrer von zweirädrigen Fahrzeugen.

Es wird eine Technologie benötigt, die sowohl den gefährdeten Verkehrsteilnehmer als auch den Autofahrer warnt, wenn ein Unfall zwischen ihnen wahrscheinlich ist.  Das ist keine leichte Herausforderung, denn die Warnungen müssen rechtzeitig verschickt werden und es muss auch sichergestellt werden, dass keine Personen gewarnt werden, die nicht wirklich vom herannahenden Auto betroffen sind.

Um dieses Ziel zu erreichen, werden drahtlose Netzwerke, GPS und Sensorik eingesetzt.

Die Ergebnisse zeigen, dass die Zahl der Unfälle mit gefährdeten Verkehrsteilnehmern drastisch gesenkt werden kann.

Deshalb wird viel mehr Arbeit in dieses Thema gesteckt, denn es zeigt sich bereits, dass die Technologien viele Verkehrsunfälle potenziell verhindern können.

\end{otherlanguage*}
\acresetall

\cleardoublepage
\tableofcontents
\TODO{The table of contents should fit on one page. When in doubt, adjust the \texttt{tocdepth} counter.}

\cleardoublepage
\pagenumbering{arabic}


\chapter{Introduction}
\label{chap:introduction}

\TODO{umstrukturieren, vielleicht doch keine sektionen!}

\section{Motivation}
\label{sec:motivation}

Within the last century the usage of motorized vehicles has grown rapidly. While this fact brings the great benefit of full mobility, it does not come without its downsides. One of them is the high number of traffic accidents. According to~\cite{v2pcomm} in roughly half of all traffic accidents, \acp{VU} are involved. Hence, there is huge potential to prevent injuries or even deaths of many people. This thesis will give you an overview of the current state of research in this area. While there are many approaches beyond the discipline of Wireless Networking (e.g. Computer Vision for pedestrian recognition), this thesis will focus mainly on how to prevent traffic accidents by using wireless network technology. However, the other areas will also be explained briefly since they often work hand in hand with Wireless Networking.

\section{Structure of the Thesis}
\label{sec:structure}

The thesis is structured as follows. First, in chapter~\ref{chap:v2p} we will look at so called \acp{V2PCS} which use wireless networks in order for pedestrian and vehicles to communicate so that accidents between them can be prevented.

Second, in chapter~\ref{chap:perception} we will look at detecting \acp{VU} by using perception of cars. We will only shortly dive into this topic, because it is not directly connected to Wireless Networking. Nevertheless, it is necessary to know the basics in order to understand chapter~\ref{chap:fusion}.

There, a combination of \acp{V2PCS} and perception will be used to improve the protection of \acp{VU}.

Finally, in chapter~\ref{chap:conclusion} all the approaches will be summarized and also we will see what future tasks are still left in this area to make it usable in the real world.


\chapter{V2P Communication Systems}
\label{chap:v2p}

The basic idea behind \acp{V2PCS} is that a driving vehicle (most often a car) and a \ac{VU} (most often a pedestrian) exchange messages over a wireless network in order tell each other their positions. By doing that, accidents between them can be prevented because both the car driver and the pedestrian can react to the received messages.

There are several approaches to implements such a system and they will be discovered in this chapter.

\section{With special Device}\label{sec:special device}

First, we will look at such communication systems which use a special device for the \ac{VU} in order to send his messages. In \cite{v2pprotection} and \cite{watchover} such systems have been proposed. 

The device from \cite{v2pprotection} can be seen in the bottom of figure~\ref{fig:device}. In the top of figure~\ref{fig:device} you can see the warning unit that will be installed in the car.

\begin{figure}[h]
	\centering
	\includegraphics[width=0.3\textwidth]{figures/1_device}
	\caption{The Warning Unit for Cars and the Warning Device for Pedestrians from \cite{v2pprotection}}%
	\label{fig:device}%
\end{figure}

The task of the device for the pedestrian is to periodically send \enquote{Hello}-Messages including the pedestrian's position so that the warning unit in the car can receive those messages and detect a probably dangerous situation. Then the warning unit sets itself into a warning status (which can be seen in figure~\ref{fig:device}).

It is very important that the car driver does not get too many warning messages because otherwise he would loose sensibility for the warning. Therefore, a key issue is to ensure that only warning are created if the distance between car and pedestrian decreases gradually.

Figure~\ref{fig:chart1} shows the ratio of meaningful warnings divided by the number of all warning which is roughly 50\% because in this test scenario only one pedestrian is present. That ratio is not satisfying, but is only because of the low number of pedestrians.

In figure~\ref{fig:chart2} you see the same statistics, but for a scenario with 10 pedestrians. Here the same ratio is almost 100\%.

\TODO{hier fehlt noch warum die ratios so sind}

\begin{figure}[h]
	\centering
	\includegraphics[width=0.8\textwidth]{figures/2_chart}
	\caption{Warning statistics in a test scenario with one pedestrian from \cite{v2pprotection}}%
	\label{fig:chart1}%
\end{figure}

\begin{figure}[h]
	\centering
	\includegraphics[width=0.8\textwidth]{figures/3_chart}
	\caption{Warning statistics in a test scenario with 10 pedestrians from \cite{v2pprotection}}%
	\label{fig:chart2}%
\end{figure}

\TODO{grafiken noch genauer erklären, vor allem mit der distanz}

\section{Using Smart Phones}\label{sec:smartphone}

In \cite{v2pcomm}, another approach has been developed which is based on the same idea as \cite{v2pprotection}, but it is more detailed. Moreover, it has the great benefit that no additional device for the \ac{VU} is needed because in this system a smart phone or tablet will be used as the warning device for the pedestrians.

The approach is more detailed, because it formulates precise requirements that an implementation of a \ac{V2PCS} should have in order to be able to work in reality. The  requirements basically boil down to the following: One has to calculate the distance at which the warning messages are to be delivered. In section~\ref{sec:special device} we have already seen that this distance should not be too big, because otherwise too many meaningless warning would be received. But accordingly, the distance must not be too small, because the pedestrian and the car driver must have enough time to react to the messages.

To calculate this distance, the authors of \cite{v2pcomm} have taken into consideration many parameters, e.g. the velocity of the car, the reaction time of the pedestrian, the expected positioning error of GPS (which is approximately 10m), the expected packet transmission time of Wi-Fi and so on. As an example, we take a look at equation~\ref{equi:minDistance} where the lower bound $d_{min}$ of the distance is written down.
\begin{align}
d_{min}=v \times (t_p+t_r+t_{tx})+gnss_{err_{car}}+gnss_{err_{ped}}\label{equi:minDistance}
\end{align}
Here $v$ is the car's velocity. Moreover, $t_p$ is the time for perception of the pedestrian, $t_r$ is the reaction time of the pedestrian, and $t_{tx}$ is the packet transmission delay. $gnss_{err_{car}}$ and $gnss_{err_{ped}}$ are the positioning errors of the car and the pedestrian, respectively.

For simplification, we will not look in every detail of this calculation as it is very detailed. But in principle it is crucial to computer a lower and upper bound for the distance at which the warning messages should be sent. Because you can then check whether your system always fulfills these bounds which the system from \cite{v2pcomm} does.

Finally, the distance calculated is 39.5, 52.3, and 72.0 m, when the velocity of the car is 30, 50, 80 km/h, respectively. \cite{v2pcomm}
\\\\
Another important concept in this context of when to send warning messages is the so called \ac{GDA}. You can understand the concept by considering figure~\ref{fig:gda}. In (a) a car is approaching to pedestrians, P1 and P2. As the car is yet far away from P1 and P2 it is possible that both P1 and P2 could be involved in an accidents with the car, therefore both of them are in the \ac{GDA}.

In (b) the car has already come nearer to P1 and P2 and is currently turning left. Due to that fact, it is no longer possible for the car to approach the position of P2. So, P2 is no longer in the \ac{GDA} of the car and P2 will not receive a warning message because of the car, only P2 will get one.

\begin{figure}[h]
	\centering
	\includegraphics[width=0.8\textwidth]{figures/5_gda}
	\caption{The Geographical Destination Area of a moving car at two different points in time, from \cite{v2pcomm}}%
	\label{fig:gda}%
\end{figure}

The \ac{GDA} of a car will be computed periodically and this makes sure that even less meaningless warning messages will be sent. The \ac{GDA} can be computed by taking several parameters of the car (e.g. the yaw rate) into account.

\TODO{jetzt erwähnen dass wlan auch lange dauert und so auf nächste sektion hinleiten}

\section{Faster WiFi Message Exchange by using Beacon Stuffing}\label{sec:beacon}

The principle from chapter~\ref{sec:smartphone} has one disadvantage that has not yet been emphasized.
Typically, when two devices want to communicate with each other over WiFi, they have to establish a WiFi connection first. This process takes time \TODO{WIE VIEL ZEIT??}. But there is an approach from the authors of \cite{beacon} to avoid that connection establishing phase by using WiFi Beacon Stuffing. Beacons are usually used in order to propagate the existence of a WiFi network. They are broadcasted periodically by the WiFi access point and contain information about the network (e.g. the SSID).

In \cite{beacon} the SSID field is replaced by a so called WiFi Honk Information Packet, which includes information about the moving car, namely the latitude, longitude, speed, and direction. \acp{VU} or other cars then receive this information and compute the likelihood of an accident.

Figure~\ref{fig:beacon} shows the results of \cite{beacon} compared to WiFi without Beacon Stuffing. Sub figure (a) shows that pedestrians have a little more time to react, but obviously if the velocity of the car comes close to 100 mph there is almost no time to react. Sub figure (b) shows the probability of collision on which Beacon Stuffing has a decisive impact. Whereas the plot normal WiFi has an exponential growth, the growth of the plot for WiFi Honk (Beacon Stuffing) is only linear. Therefore, warning messages can be sent more aggressively. \TODO{WARUM???}

\begin{figure}[h]
	\centering
	\includegraphics[width=1\textwidth]{figures/6_beacon}
	\caption{Evaluation of Beacon Stuffing, from \cite{beacon}. \\POC stands for \enquote{Probability of Collision}, TAS stands for \enquote{Time available to stop}.}%
	\label{fig:beacon}%
\end{figure}

\chapter{Detecting Pedestrians by using Perception}
\label{chap:perception}

\TODO{Hier kommt nur ein kleiner Ausblick hin, da es nicht direkt etwas mit Wireless Networking zu tun hat}

\chapter{Fusion of Perception and V2P Communication}\label{chap:fusion}




\chapter{Conclusion}
\label{chap:conclusion}


\begin{itemize}
\item summarize again what your paper did, but now emphasize more the results, and comparisons
\item write conclusions that can be drawn from the results found and the discussion presented in the paper
\item future work (be very brief, explain what, but not much how, do not speculate about results or impact)
\item recommended length: one page.
\end{itemize}



\cleardoublepage

\listofabbreviations
\clearpage

\listoffigures
\clearpage


\printbibliography

\end{document}
